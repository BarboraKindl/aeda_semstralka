\documentclass{article}
\usepackage{graphicx} % Required for inserting images
\usepackage[utf8]{inputenc}   % Podpora UTF-8
\usepackage[T1]{fontenc}      % T1 kódovanie
\usepackage[english,czech,slovak]{babel} % Český/slovenský jazyk
\usepackage{graphicx}         % Na vkladanie obrázkov
\usepackage{microtype}        % Lepšie zarovnanie textu

\title{Pohyby očí a porucha chování v REM spánku}
\author{Bc. Barbora Kindlová, Bc. Milan Španko, Bc. Jakub Benetin}
\date{November 2024}

\begin{document}

\maketitle

\section{Úvod}

Oční pohyby jsou komplexní neurologický proces zahrnující řadu mozkových struktur, včetně neokortexu, bazálních ganglií, mozkového kmene a mozečku. Tyto struktury spolupracují na řízení aspektů pohybů, jako je reakční čas, přesnost a rychlost. Analýza očních pohybů může odhalit důležité informace o funkci centrální nervové soustavy, lokalizovat postižené oblasti a poskytnout poznatky o patofyziologii neurodegenerativních onemocnění, například Parkinsonovy nemoci (PD), kterou lze detekovat na základě známých specifických abnormalit spojených s dysfunkcí bazálních ganglií. \cite{pretegiani2017eye}

Vyšetření očních pohybů je slibnou metodou pro zkoumání neurodegenerativních změn v mozku. Je neinvazivní, rychlé a cenově dostupnější než standardní hodnotící metody, což umožňuje zpracování větších vzorků dat. Moderní technologie sledování očí poskytují přesné kvantitativní údaje, snižují variabilitu i subjektivitu dat a zvyšují diagnostickou hodnotu.\cite{sekar2024detecting}

Parkinsonova nemoc (PN) je nejčastější neurodegenerativní onemocnění pohybového systému, které se projevuje trasem, ztuhlostí, zpomalením pohybů a poruchami rovnováhy. Příčinou je ztráta dopaminergních neuronů v substantia nigra a akumulace proteinu alfa-synukleinu tvořícího Lewyho tělíska. Příčina bývá neznámá, genetické faktory se podílejí na 5–10\% případů. Léčba zmírňuje příznaky, avšak nezastavuje progresi. Prevalence roste s věkem a riziko zvyšují také environmentální faktory.\cite{balestrino2020parkinson}

Porucha chování v REM spánku (RBD) je klinická entita charakterizovaná absencí normální atonie během REM fáze spánku. Pacienti často „prožívají“ své sny pohybem, mluvením nebo křikem. RBD je spojeno s více REM spánkem, periodickými pohyby nohou a zhoršením kognice. Více než 50\% pacientů s RBD postupně rozvine neurodegenerativní onemocnění, nejčastěji Parkinsonovu nemoc. RBD tak umožňuje studium prodromálních fází neurodegenerace před výskytem motorických příznaků PD.\cite{jiang2017rbd}

Projekt se zaměřuje na analýzu očních pohybů jako metody pro lepší pochopení souvislostí mezi těmito poruchami. Sleduje charakteristiky pohybů, jako je reakční čas, rychlost a přesnost, a jejich vztah k raným stadiím neurodegenerace. Tento přístup by mohl přispět k časnější diagnostice a efektivnějšímu terapeutickému přístupu.

Cílem projektu je prozkoumat rozdíly v parametrech očních pohybů mezi zdravými jedinci (HC) a pacienty s RBD či PD a určit, zda lze tyto parametry použít pro predikci či lepší pochopení neurodegenerativních procesů. Zaměříme se na tři hlavní otázky:
\begin{enumerate}
    \item Jsou mezi těmito skupinami významné rozdíly v parametrech očních pohybů?
    \item Existuje vztah mezi parametry očních pohybů a klinickými měřítky, jako je skóre UPDRS III u pacientů s PD nebo MoCA u pacientů s RBD?
    \item Mají parametry očních pohybů prediktivní potenciál při odhalování prodromálních příznaků neurodegenerace?
\end{enumerate}

\section{Metody}
\subsection{Deskriptivní data}
\subsection{popis features}
\subsection{Statistická analýza}
\section{Výsledky}

suše napsat co nám vyšlo
\section{Diskuze}
\subsection{Rozdíly mezi skupinami v parametrech}
Statistická analýza první hypotézy ukázala významné rozdíly v parametrech očních pohybů mezi skupinami HC (zdravé kontroly), RBD (porucha chování během REM spánku) a PD (Parkinsonova choroba).

U RBD byly abnormality zaznamenány především v reakčních časech při anti- a prosakádách, což poukazuje na počínající kognitivní a motorické zhoršení v prodromální fázi neurodegenerace. PD vykazovalo výrazné odchylky oproti HC ve všech sledovaných parametrech, což bylo očekávané. Parametry očních pohybů, jako reakční čas, rychlost či přesnost, se jeví jako potenciální biomarkery různých stádií neurodegenerace.

Analýza signifikantních rozdílů mezi skupinami na zaklade parametrů z vícero kategorií přinesla tyto výsledky:
\begin{itemize}

\item \textbf{PCA1 (horizontální sakády):} Rozdíly HC vs. PD (p = 0.048) a PD vs. RBD (p = 0.046) reflektují progresi motorických a kognitivních změn.
\item \textbf{PCA2 (horizontální sakády):} Rozdíly HC vs. PD (p = 0.031) a HC vs. RBD (p = 0.002) poukazují na abnormality již v RBD.
\item \textbf{Reakční časy a věk:} Klíčové rozdíly HC vs. RBD (PCA1, p = 0.038) a PD vs. RBD (PCA2, p = 0.007).
\item Vertikální sakády se neprokázaly jako spolehlivý prediktor, možná však jen na základě volby datasetu.
\end{itemize}


\section{Závěr}

\section{Random poznámky}
\begin{itemize}
    \item co chceme řešit - jasně zadefinujeme - statistické testy
    \item testy - výsledky (nemáme použít milion testů) - zamyslet se nad našim GOALEM
    \item odevzdáváme pdf - bez kodu
    \item sdělit na co jsme přišli
    \item odevzádávme je, ale prostře hlavní je to pdf
    \item \textbf{pět A max}
\end{itemize}

\bibliographystyle{plain}
\bibliography{mybib}

\end{document}
